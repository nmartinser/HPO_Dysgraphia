\section{Discusión}

Tras observar los genes con los que se relacionan y las comunidades que forman (tabla \ref{tabla:enrique200}), resalta que la mayoría de ellos participan en actividades relacionadas con la reparación, formación y procesos metabólicos del ADN.\\

Además, si nos fijamos en estudios previamente realizados en torno a comunidades compuestas por los genes en cuestión, se desarrollan en este mismo contexto de reparación genómico. El artículo de Mota, M.B.S. et al. \cite{Mota2019} relaciona genes que aparecen en nuestro grafo, como H2AX, TP53BP1, RAD51 y EXO1, con la reparación de roturas de doble cadena, que es una de las principales vías de reparación.\\


Por otro lado, los HPOs mayoritariamente relacionados con la disgrafía también forman parte de trastornos mentales o psicológicos como la depresión, ansiedad o problemas en la sincronización de la marcha (tabla \ref{tab:dysgraphia-relaciones}). De esta manera podemos centralizar que efectivamente es una afección cerebral y no motora muscular.\\

Sin embargo, lo anteriormente explicado no implica un gran descubrimiento acerca de este fenotipo, las primeras relaciones obtenidas en el estudio resultan ser ya conocidas u obvias. Es por ello que investigando más profundamente y buscando patrones comunes en los archivos de enriquecimiento, hemos encontrado dos relaciones comunes menos intuitivas y por tanto más significativas.\\

Como se ha mencionado anteriormente, muchos de los genes relacionados con disgrafía se encargan de la reparacion del ADN. Estos genes desempeñan un papel importante en el mantenimiento de la integridad genómica y en la protección contra el desarrollo del cáncer. Cuando la primera etapa de eliminar el zonas dañadas en el ADN es más activa que la última etapa de volver a sintetizar el ADN durante el proceso de reparación, este desequilibrio provoca una acumulación excesiva de productos intermedios de ADN sin reparar, lo cual a su vez aumenta el riesgo de mutaciones y cáncer. \cite{Song2012}.\\

Además, estudios relacionan la deficiencia en la reparación del daño al ADN nuclear con varios trastornos neurodegenerativos \cite{Jeppesen2011}. Gracias al enriquecimento del grafo ampliado podemos comprobar que la mayoría de los genes implicados en ese grafo se encuentran en el núcleo de la célula y se encargan de la reparación de daños.  Si los mecanismos de reparación del ADN no funcionan eficientemente, es más probable que se acumulen mutaciones y daños genéticos en las células, lo que podría contribuir al desarrollo de trastornos neurodegenerativos, como disgrafía\\

Fijando la categoría a 'TISSUES' en el enriquecimiento de los 200 genes, aparecen muchas entradas con descripción derivada del aparato reproductor femenino, así como el desarrollo embrionario: \textit{ Cervical carcinoma cell, Embryo, Female reproductive system, Uterus... }

\begin{table}[h]
	\centering
	\caption{Datos sobre la categoría 'TISSUES' en el enriquecimiento de los 200 genes.}
	\label{tab:datos-tissues}
	\begin{tabular}{|c|c|c|c|c|}
		\hline
		\textbf{Categoría} & \textbf{Nº de genes} & \textbf{p-value} & \textbf{FDR} & \textbf{Descripción} \\
		\hline
		TISSUES & 35 & 1.43e-24 & 3.47e-21 & Cervical carcinoma cell \\
		TISSUES & 48 & 4.1e-18 & 4.97e-15 & Leukemia cell \\
		TISSUES & 31 & 2.34e-17 & 1.89e-14 & Lymphocytic leukemia cell \\
		TISSUES & 49 & 2.15e-16 & 1.3e-13 & Blood cancer cell \\
		TISSUES & 39 & 2.01e-15 & 9.74e-13 & Embryo \\
		TISSUES & 22 & 4.33e-15 & 1.75e-12 & Pronephros \\
		TISSUES & 22 & 1.83e-09 & 6.36e-07 & Bone marrow cancer cell \\
		TISSUES & 16 & 2.55e-09 & 7.73e-07 & Chronic lymphocytic leukemia cell \\
		TISSUES & 101 & 9.29e-09 & 2.5e-06 & Female reproductive system \\
		TISSUES & 56 & 1.43e-08 & 3.46e-06 & Organism form \\
		TISSUES & 5 & 2.36e-08 & 5.21e-06 & U2-OS cell \\
		TISSUES & 53 & 2.42e-08 & 5.21e-06 & Embryonic structure \\
		TISSUES & 103 & 3.74e-08 & 6.97e-06 & Reproductive system \\
		\hline
	\end{tabular}
	
\end{table}

Esta segmentación puede ser muy significativa dando pie a hipótesis sobre si el origen de la disgrafía está relacionada con el desarrollo embrionario o con la concepción. Además, se puede hilar a la alta interacción de estos genes con afecciones cancerígenas puesto que aparecen entradas sobre cáncer ovárico en el enriquecimiento \cite{Mad2014}.\\

En general, se han encontrado varias posibles relaciones de la disgrafía con otras patologías, fenotipos y tejidos. De esta manera logramos ampliar el conocimiento sobre este HPO y focalizar su área de efecto para proponer tratamientos parecidos a estas relaciones resultantes.

