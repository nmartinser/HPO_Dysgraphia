\section{Introducción}
La escritura es una habilidad que se desarrolla en la infancia, estamos rodeados de textos que leer y que implican nuestro día a día. La disgrafia es un trastorno de aprendizaje que surge en esta etapa del desarrollo que afecta a las habilidades de escritura. Esto puede involucrar dificultades de cualquier nivel: caligrafía, escritura lenta, ortografía...\\

Como tratamiento para el manejo de la disgrafia, se llevan a cabo intervenciones organizadas en tres categorías: acomodación, modificación y revalorización. Las acomodaciones incluyen estrategias como proporcionar instrumentos de escritura especiales y permitir el uso de grabadoras y correctores ortográficos. Las modificaciones implican ajustar las expectativas académicas, dividiendo tareas extensas o permitiendo alternativas como informes orales. La revalorización se basa en un enfoque de respuesta a la intervención, es decir, un cálculo continuo del estado de su disgrafia implica evaluar y proporcionar apoyo específico según las dificultades del individuo. Las intervenciones pueden centrarse en tareas motoras, ortográficas y habilidades de escritura superiores. Los enfoques combinados son efectivos, y las intervenciones tempranas son cruciales. Además, las tecnologías como el reconocimiento de voz y las computadoras pueden ayudar, especialmente si la automaticidad de la escritura es un desafío. \cite{Chung2015}