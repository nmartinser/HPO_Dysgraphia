\documentclass{bmcart}

%%%%%%%%%%%%%%%%%%%%%%%%%%%%%%%%%%%%%%%%%%%%%%
%%                                          %%
%% CARGA DE PAQUETES DE LATEX               %%
%%                                          %%
%%%%%%%%%%%%%%%%%%%%%%%%%%%%%%%%%%%%%%%%%%%%%%

%%% Load packages
\usepackage{amsthm,amsmath}
\usepackage{graphicx}
%\RequirePackage[numbers]{natbib}
%\RequirePackage{hyperref}
\usepackage[utf8]{inputenc} %unicode support
%\usepackage[applemac]{inputenc} %applemac support if unicode package fails
%\usepackage[latin1]{inputenc} %UNIX support if unicode package fails
\usepackage{hyperref} % para añadir URL y referencias


%%%%%%%%%%%%%%%%%%%%%%%%%%%%%%%%%%%%%%%%%%%%%%
%%                                          %%
%% COMIENZO DEL DOCUMENTO                   %%
%%                                          %%
%%%%%%%%%%%%%%%%%%%%%%%%%%%%%%%%%%%%%%%%%%%%%%

\begin{document}

	\begin{frontmatter}
	
		\begin{fmbox}
			\dochead{Research}
			
			%%%%%%%%%%%%%%%%%%%%%%%%%%%%%%%%%%%%%%%%%%%%%%
			%% INTRODUCIR TITULO PROYECTO               %%
			%%%%%%%%%%%%%%%%%%%%%%%%%%%%%%%%%%%%%%%%%%%%%%
			
			\title{Estudio del fenotipo Disgrafía}
			
			%%%%%%%%%%%%%%%%%%%%%%%%%%%%%%%%%%%%%%%%%%%%%%
			%% AUTORES. METER UNA ENTRADA AUTHOR        %%
			%% POR PERSONA                              %%
			%%%%%%%%%%%%%%%%%%%%%%%%%%%%%%%%%%%%%%%%%%%%%%
			
			\author[
			  addressref={aff1},                   % ESTA LINEA SE COPIA IGUAL PARA CADA AUTOR
			  corref={aff1},                       % ESTA LINEA SOLO DEBE TENERLA EL COORDINADOR DEL GRUPO
			  email={nmartins@uma.es}   % VUESTRO CORREO ACTIVO
			]{\inits{N.M.S.}\fnm{Nerea} \snm{Martín Serrano}} % inits: INICIALES DE AUTOR, fnm: NOMBRE DE AUTOR, snm: APELLIDOS DE AUTOR
			\author[
			  addressref={aff1},
			  email={belopcarlos@uma.es}
			]{\inits{C.B.L}\fnm{Carlos} \snm{Beltrán López}}
			
			\author[
			addressref={aff1},
			email={?@uma.es}
			]{\inits{D.C.T}\fnm{David} \snm{Cubillos del Toro}}
			
			
			%%%%%%%%%%%%%%%%%%%%%%%%%%%%%%%%%%%%%%%%%%%%%%
			%% AFILIACION. NO TOCAR                     %%
			%%%%%%%%%%%%%%%%%%%%%%%%%%%%%%%%%%%%%%%%%%%%%%
			
			\address[id=aff1]{%                           % unique id
			  \orgdiv{ETSI Informática},             % department, if any
			  \orgname{Universidad de Málaga},          % university, etc
			  \city{Málaga},                              % city
			  \cny{España}                                    % country
			}
		
		\end{fmbox}% comment this for two column layout
		
		\begin{abstractbox}
		
			\begin{abstract} % abstract
			
			%%%%%%%%%%%%%%%%%%%%%%%%%%%%%%%%%%%%%%%%%%%%%%%
			%% RESUMEN BREVE DE NO MAS DE 100 PALABRAS   %%
			%%%%%%%%%%%%%%%%%%%%%%%%%%%%%%%%%%%%%%%%%%%%%%%	
	
			\textbf{1. Introducción}
			La escritura, una habilidad crucial en la infancia, puede verse afectada por la disgrafía, un trastorno de aprendizaje que impacta la escritura, caligrafía, ortografía y velocidad de escritura. Este trastorno puede perjudicar el rendimiento escolar y requiere intervenciones específicas. El proyecto tiene como meta identificar relaciones funcionales y estructurales entre los genes del HPO para definir fenotipos derivados, estudiar zonas anatómicas críticas y explorar enfermedades específicas asociadas a la patología.
			
			\textbf{2. Materiales y Métodos}
			Se utilizaron bases de datos como Human Phenotype Ontology (HPO) y STRING-DB. Se realizaron análisis de redes, detección de comunidades y enriquecimiento funcional.
			
			\textbf{3. Resultados}
			El grafo bipartito reveló relaciones entre la disgrafía y términos HPO como "Dificultad de Marcha" y "Depresión". La red de genes mostró interacciones clave, y la propagación de la red reveló genes adicionales relacionados con la disgrafía. El enriquecimiento funcional reveló asociaciones con procesos de reparación del ADN y desarrollo embrionario.
			
			\textbf{4. Discusión}
			Se confirmaron conexiones conocidas, pero se destacaron relaciones menos intuitivas, como la influencia en la reparación del ADN y posibles vínculos con el desarrollo embrionario. Estos hallazgos sugieren la importancia de investigar más a fondo para comprender mejor los mecanismos subyacentes de la disgrafía y su relación con otras patologías. Este estudio proporciona una base para futuras investigaciones y enfoques terapéuticos.
			
			\end{abstract}
			
			%%%%%%%%%%%%%%%%%%%%%%%%%%%%%%%%%%%%%%%%%%%%%%
			%% PALABRAS CLAVE DEL PROYECTO              %%
			%%%%%%%%%%%%%%%%%%%%%%%%%%%%%%%%%%%%%%%%%%%%%%
			
			\begin{keyword}
			\kwd{Disgrafía}
			\kwd{NetworkX}
			\kwd{STRING}
			\kwd{HPO}
			\end{keyword}
		
		
		\end{abstractbox}
	
	\end{frontmatter}
	
	
	%%%%%%%%%%%%%%%%%%%%%%%%%%%%%%%%%
	%% COMIENZO DEL DOCUMENTO REAL %%
	%%%%%%%%%%%%%%%%%%%%%%%%%%%%%%%%%
	
	\section{Introducción}
La escritura es una habilidad que se desarrolla en la infancia, estamos rodeados de textos que leer y que implican nuestro día a día. La disgrafia es un trastorno de aprendizaje que surge en esta etapa del desarrollo que afecta a las habilidades de escritura \cite{Chung2015}. Puede manifestarse de diversas maneras, reflejando problemas en la memoria ortográfica a largo plazo, el proceso de conversión de sonido a escritura y, en algunos casos, la memoria de trabajo ortográfica \cite{ McCloskey2017}. Esto puede involucrar dificultades de cualquier nivel: caligrafía, escritura lenta, ortografía...\\

Hay múltiples tipos de deterioro que pueden ser la causa fundamental de la disgrafia.  La causa más comúnmente propuesta es un déficit en el procesamiento fonológico, lo que dificultaría la comprensión de las relaciones entre sonidos y grafías en la escritura. Sin embargo, también se han observado casos de disgrafía con habilidades fonológicas intactas. Otras posibles causas distantes incluyen déficits en el dominio visual y problemas de control motor \cite{ McCloskey2017}.\\

La disgrafía a menudo se asocia con otros trastornos como la dislexia. A nivel neurofisiológico, estos trastornos parecen compartir áreas cerebrales similares. La disgrafía también comparte similitudes con el trastorno del desarrollo de la coordinación y las dificultades de aprendizaje en el lenguaje oral y escrito \cite{Marek2020}. \\

La disgrafía puede tener un impacto negativo en el rendimiento escolar de los niños. Muchos niños con disgrafía no pueden organizar coherentemente sus pensamientos en papel o escribir de manera legible. Esta discapacidad debe ser reconocida y tratada antes de que genere consecuencias negativas duraderas para el niño. \cite{ Crouch2007} \\



Como tratamiento para el manejo de la disgrafia, se llevan a cabo intervenciones organizadas en tres categorías: acomodación, modificación y revalorización. Las acomodaciones incluyen estrategias como proporcionar instrumentos de escritura especiales y permitir el uso de grabadoras y correctores ortográficos. Las modificaciones implican ajustar las expectativas académicas, dividiendo tareas extensas o permitiendo alternativas como informes orales. La revalorización se basa en un enfoque de respuesta a la intervención, es decir, un cálculo continuo del estado de su disgrafia implica evaluar y proporcionar apoyo específico según las dificultades del individuo. Las intervenciones pueden centrarse en tareas motoras, ortográficas y habilidades de escritura superiores. Los enfoques combinados son efectivos, y las intervenciones tempranas son cruciales. Además, las tecnologías como el reconocimiento de voz y las computadoras pueden ayudar, especialmente si la automaticidad de la escritura es un desafío. \cite{Chung2015}
	\section{Materiales y métodos}

En esta sección, describiremos la metodología utilizada en el estudio de la Disgrafia,
junto con los materiales empleados. La metodología se dividió en varias etapas, las
cuales se detallarán a lo largo de esta sección y se pueden observar en la imagen \ref{fig:workflow}.

\begin{figure}[h!]
	\includegraphics[width=0.9\textwidth]{figures/workflow.JPG}
	\caption{Flujo de trabajo}
	\label{fig:workflow}
\end{figure}

\subsection{Datos biológicos}

Lo primero que se realizó fue buscar el fenotipo Disgrafía en la Human Phenotype Ontology \cite{HPO2021}, conociendo que su identificador es HP:0010526. De esta base de datos, obtuvimos un archivo tabulado que enumera para cada gen las clases HPO más específicas. Las primeras cinco filas se pueden visualizar en la tabla \ref{tabla:geneshpo}.

\begin{table}[h]
	\centering
	\caption{Cabecera del archivo}
	\label{tabla:geneshpo}
	\resizebox{1.1\textwidth}{!}{
		\begin{tabular}{|c|c|c|c|c|c|c|}
			\hline
			Gene id (ncbi) & Gene symbol & HPO id & HPO name & frequency & Disease id \\
			\hline
			10 & NAT2 & HP:0000007 & Autosomal recessive inheritance & - & OMIM:243400 \\
			10 & NAT2 & HP:0001939 & Abnormality of metabolism/homeostasis & - & OMIM:243400 \\
			16 & AARS1 & HP:0002460 & Distal muscle weakness & 15/15 & OMIM:613287 \\
			16 & AARS1 & HP:0002451 & Limb dystonia & 3/3 & OMIM:616339 \\
			16 & AARS1 & HP:0008619 & Bilateral sensorineural hearing impairment & HP:0040283 & ORPHA:33364 \\
			\hline
		\end{tabular}
	}
\end{table}

La tabla \ref{tabla:geneshpo} proporciona el identificador de gen NCBI, el símbolo del gen, el identificador HPO y el nombre del término. Si está disponible, se muestra la frecuencia. Por ejemplo, la mutación en el gen AARS1 causa \textit{leucoencefalopatía}. La frecuencia del término HPO Ataxia sensorial esta anotada como 1 de 2 debido a la información en Sundal C, et al. \cite{Sundal2019}. La última columna muestra anotaciones realizadas por el equipo HPO (utilizando identificadores de enfermedades de OMIM), así como anotaciones proporcionadas por el equipo de Orphanet (utilizando identificadores de enfermedades de ORPHA).

\subsection{Grafo bipartito}


En un grafo bipartito, los vértices se organizan en dos conjuntos distintos, de modo que cada arista conecta un vértice de un conjunto con otro del segundo conjunto. En términos más simples, no existen aristas que conecten vértices dentro del mismo conjunto. En nuestro contexto, los conjuntos de vértices representan genes y términos HPO. De esta manera, obtenemos un grafo bipartito que conecta distintos términos HPO al nuestro, a través de genes. 

Para llevar a cabo esta representación y conexión entre genes y términos HPO, hemos utilizado la librería de Python NetworkX \cite{BookNetworkX}. 
	
\section{Resultados}

\graphicspath{ {../results/} }

En esta sección se van a presentar los resultados obtenidos tras realizar el estudio del fenotipo Disgrafía.

\subsection{Red de genes asociados}

Después de descargar la red de genes asociados a la Disgrafía de la HPO, observamos que contenía 51 genes. Posteriormente, obtuvimos una red de genes relacionados con la Disgrafía utilizando la base de datos STRING-DB, la cual se puede observar en la imagen \ref{fig:genesAsociados}.

\begin{figure}[h!]
	\centering
%	\includegraphics[width=0.7\textwidth]{stringdb_51_genes.png}
	\caption{Red de genes asociados}
	\label{fig:genesAsociados}
\end{figure}

Se aplicó el algoritmo Diamond para propagar la red de genes y descubrir potenciales genes adicionales relacionados con la Disgrafía. El resultado de este proceso se encuentra en el archivo "propagated\_genes.txt", que contiene un listado con 200 genes. Utilizando estos genes, se creó una red mediante la librería de String. La cabecera del archivo que contiene la red se puede visualizar en la tabla \ref{tabla:resultDiamond} (se han omitido algunas columnas del archivo para mejor visualización).

\begin{table}[h]
	\centering
	\caption{Red de genes con los resultados de Diamond}
	\label{tabla:resultDiamond}
	\begin{tabular}{|c|c|c|c|c|c|}
		\hline
		stringId\_A & stringId\_B & preferredName\_A & preferredName\_B & score \\
		\hline
		9606.ENSP00000013807 & 9606.ENSP00000265433 & ERCC1 & NBN & 0.7 \\
		9606.ENSP00000013807 & 9606.ENSP00000347232 & ERCC1 & BLM & 0.701 \\
		9606.ENSP00000013807 & 9606.ENSP00000494957 & ERCC1 & UBE2T & 0.702 \\
		9606.ENSP00000013807 & 9606.ENSP00000229769 & ERCC1 & FANCE & 0.71 \\
		\hline
	\end{tabular}
\end{table}

Cada fila de esta red contiene información sobre la interacción de dos proteínas. Las cuatro primeras columnas incluyen los identificadores de los dos genes que están interactuando (tanto el StringId como el nombre). La cuarta columna (\textit{score}) es la puntuación combinada, la cual ya se mencionó en la sección \ref{section:redProteinas}.


\subsection{Detección de comunidades}

Se realizó un análisis de detección de comunidades en la red ampliada de genes. El algoritmo greedy\_modularity\_communities de NetworkX se empleó para identificar conjuntos de genes más densamente conectados entre sí. Los resultados de este análisis se presentan en la imagen \ref{fig:comunidades}.

\begin{figure}[h!]
	\centering
	\includegraphics[width=0.7\textwidth]{graph_communities.png}
	\caption{Comunidades de la red de genes}
	\label{fig:comunidades}
\end{figure}

Este algoritmo ha detectado tres comunidades. Cada color de la imagen \ref{fig:comunidades} representa una comunidad distinta.

\subsection{Enriquecimiento}

Se llevó a cabo un análisis de enriquecimiento funcional para identificar asociaciones significativas entre los genes y funciones específicas en procesos biológicos y rutas metabólicas. Este análisis se realizó tanto en el grafo ampliado como en cada una de las comunidades detectadas en la red de genes.

Cada conjunto de resultados del enriquecimiento ha sido ordenado de manera ascendente según el false discovery rate (FDR).

\begin{itemize}
	\item Enriquecimiento grafo ampliado
\end{itemize}

	\section{Discusión}

Tras observar los genes con los que se relacionan y las comunidades que forman, resalta que la mayoría de ellos participan en actividades relacionadas con la reparación, formación y procesos metabólicos del ADN.

(IMAGEN TABLA 6)

Además, si nos fijamos en estudios previamente realizados en torno a comunidades compuestas por los genes en cuestión, se desarrollan en este mismo contexto de reparación genómico (INTRODUCIR REFERENCIAS Y RELACIONAR CON ELLAS)

Por otro lado, los HPOs mayoritariamente relacionados con la disgrafía también forman parte de trastornos mentales o psicológicos como la depresión, ansiedad o problemas en la sincronización de la marcha. De esta manera podemos centralizar que efectivamente es una afección cerebral y no motora muscular.

Sin embargo, lo anteriormente explicado no implica un gran descubrimiento acerca de este fenotipo, las primeras relaciones obtenidas en el estudio resultan ser ya conocidas u obvias. Es por ello que investigando más profundamente y buscando patrones comunes en los archivos de enriquecimiento, hemos encontrado dos relaciones comunes menos intuitivas y por tanto más significativas.

(INTRODUCIR AQUÍ MELANOMA Y TP53 ETC)

Fijando la categoría a 'TISSUES' en el enriquecimiento de los 200 genes, aparecen muchas entradas con descripción derivada del aparato reproductor femenino, así como el desarrollo embrionario: \textit{ Cervical carcinoma cell, Embryo, Female reproductive system, Uterus... }

( INTRODUCIR IMAGEN DE CLASIFICACIÓN DE 'TISSUES' )

Esta segmentación puede ser muy significativa dando pie a hipótesis sobre si el origen de la disgrafía está relacionada con el desarrollo embrionario o con la concepción. Además, se puede hilar a la alta interacción de estos genes con afecciones cancerígenas puesto que aparecen entradas sobre cáncer ovárico en el enriquecimiento.

En general, se han encontrado varias posibles relaciones de la disgrafía con otras patologías, fenotipos y tejidos. De esta manera logramos ampliar el conocimiento sobre este HPO y focalizar su área de efecto para proponer tratamientos parecidos a estas relaciones resultantes.



El objetivo de este proyecto será encontrar relaciones entre los genes que componen este HPO. Tanto funcionales como estructurales con el fin de definir fenotipos derivados, zonas anatómicas críticas interesantes de estudio o enfermedades en concreto que incluyen nuestra patología.
	
	
	%%%%%%%%%%%%%%%%%%%%%%%%%%%%%%%%%%%%%%%%%%%%%%
	%% OTRA INFORMACIÓN                         %%
	%%%%%%%%%%%%%%%%%%%%%%%%%%%%%%%%%%%%%%%%%%%%%%
	
	\begin{backmatter}
	
		\section*{Abreviaciones}%% if any
			SLD: dificultades específicas del aprendizaje
			
			HPO: Human Phenotype Ontology
			
			FDR: false discovery rate
			
			GO: Gene Ontology
		
		\section*{Disponibilidad de datos y materiales}%% if any
			Enlace al repositorio de GithHub: \url{https://github.com/nmartinser/HPO_Dysgraphia}
		
		\section*{Contribución de los autores}
			
			C.B.L: Encargado de realizar la propagación, análisis de comunidades, enriquecimiento funcional y los archivos .bash ejecutables. Escritura de introducción y discusión.
			
			N.M.S: Encargada de redactar el informe (abstract, introducción, materiales y métodos, resultados y discusión) y supervisar el código.
			 
		
		
		%%%%%%%%%%%%%%%%%%%%%%%%%%%%%%%%%%%%%%%%%%%%%%%%%%%%%%%%%%%%%%%%%%%%%%%%%%%%%%%%%%%%%%%%
		%% BIBLIOGRAFIA: no teneis que tocar nada, solo sustituir el archivo bibliography.bib %%
		%% por el que hayais generado vosotros                                                %%
		%%%%%%%%%%%%%%%%%%%%%%%%%%%%%%%%%%%%%%%%%%%%%%%%%%%%%%%%%%%%%%%%%%%%%%%%%%%%%%%%%%%%%%%%
		
		\bibliographystyle{bmc-mathphys} % Style BST file (bmc-mathphys, vancouver, spbasic).
		\bibliography{bibliography}      % Bibliography file (usually '*.bib' )
	
	\end{backmatter}
\end{document}
